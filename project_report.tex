\documentclass[a4paper, 12pt]{article}
\usepackage{array}
\setlength{\extrarowheight}{0.12cm}

\usepackage{graphics, graphicx}
\usepackage{color}
\usepackage{a4wide}
\usepackage{hyperref}
\usepackage{amsmath,amssymb}
\usepackage{anysize}
\usepackage[utf8]{inputenc}
\usepackage{listings}
\usepackage{amsfonts} 
\usepackage{graphicx} 
\usepackage{fancyhdr}
\usepackage[us,12hr]{datetime} % `us' makes \today behave as usual in TeX/LaTeX
\fancypagestyle{plain}{
\fancyhf{}
\rfoot{{\ddmmyyyydate\today} - \currenttime}

\renewcommand{\headrulewidth}{0pt}}
\pagestyle{plain}
\author{GROUP - 5} 
\title{Study Of Various Polynomial Interpolation Schemes} % change this
% change this

\date{\today} % change this

\begin{document}

%%%%%%%%%% PRELIMINARY MATERIAL %%%%%%%%%%
\maketitle
\begin{center}
\title\textbf{Authors}
\end{center}
\begin{center}
\begin{tabular}{|c|c|c|}
\hline
Name & Roll Number & Discipline\\
\hline
Abhijeet Singh Panwar & 201351005 & CS\\
Ajay Shewale & 201351030  & CS\\
Dilip Puri & 201351014 & CS\\
Rahul Nalawade & 201351017 & CS\\
Sudhanshu Jaiswal & 201352012 & IT\\
Sonu Patidar & 201351016 & CS\\
\hline
\end{tabular}
\end{center}



\begin{center}
Supervised by Dr.\ Pratik Shah. % change this
\end{center}
\newpage

\tableofcontents

\newpage
\maketitle
\begin{center}
\title{Acknowledgement of Sources}
\end{center} % this must be included in undergradate projects
For all ideas taken from other sources (books, articles, internet), the source of the ideas is mentioned in the main text and fully referenced at the end of the report.

All material which is quoted essentially word-for-word from other sources is given in quotation marks and referenced.

Pictures and diagrams copied from the internet or other sources are labelled with a reference to the web page or book, article etc.
\\[12pt]
Date \dotfill

%%%%%%%%%% MAIN TEXT STARTS HERE %%%%%%%%%%

%%%%%%%%%% SECTION %%%%%%%%%%
\newpage
\maketitle
\section{Introduction}
\hrule
$\downarrow$\\
In Engineering and Sciences,one often has a number of \textbf{data points},obtained by sampling or experimentation,which may represent the value of a function for a limited number of values of the independent variables.It is often required to \textbf{interpolate}(i.e,estimate) the value of that function for an intermidiate value of the independent variables.\\This action is accomplished by using various methods of interpolation,discussed in the following articles.
\section{Definition}\hrule
$\downarrow$\\
Interpolation is a method of contructing new data points within the range of \textbf{discrete set} of known data points.\\Since data is provided as \textbf{discrete set of points},$(x_0,y_0),(x_1,y_1),(x_2,y_2)...(x_k,y_k)$\\
A \textbf{continuous function $f(x)$} passing through these points may be used to represent these k+1 data values.\\
So now,one can find values of y for all x.\\This \textbf{interpolating function $f(x)$} can be chosen wisely according to the required result.\\
The most common interpolating function is a polynomial as it is easy to:
\begin{itemize}
\item evaluate
\item differentiate
\item integrate,etc
\end{itemize}
\newpage
\section{Polynomial interpolation}
\hrule
$\downarrow$\\
Interpolation of a given data set by a polynomial passing through every point is simply a polynomial interpolation.\\It invovles the polynomial $P(x)$ of degree n passing through given n+1 points.
\subsection{Quadratic Interpolation}
In quadratic interpolation there are given 3 points,using those we have to come up with a quadratic curve which pasess through all three points.
\\
 Suppose those three points are  \[(x_1,y_1),(x_2,y_2) \& (x_3,y_3)\] Therefore,a curve of general form \[y=a*x^2+b*x+c\] should satisfy these points to \[a*x_1^2+b*x_1+c=y_1\] \[a*x_2^2+b*x_2+c=y_2\]\[a*x_3^2+b*x_3+c=y_3\] Now the values of a , b \& c is given by

$3 \times 3$~matrix 
\[  \underbrace{\left( \begin{array}{ccc}
x_1^2 & x_1 & 1 \\
x_2^2 & x_2 & 1 \\
 x_3^2 & x_3 & 1 \end{array} \\ \right)} 
\left( \begin{array}{ccc}
a\\
b\\
c\end{array} \right)=\left(\begin{array}{ccc}
y_1\\
y_2\\
y_3\\
\end{array} \right)\]\\
From this we can deduce value of co-efficients a , b \& c by\\
\[ \left(\begin{array}{ccc}
a\\
b\\
c\end{array} \right)= x^{-1} \left(\begin{array}{ccc}
y_1\\
y_2\\
y_3\\
\end{array} \right)\] \\
\textbf{NOTE}$\rightarrow$For interpolation scheme in image processing "x" can be taken as distance function \& "y"
can be represented as the value of pixel at that position.\\ \\
\underline{\textbf{Theorem}}$\rightarrow$ Given \underline{n} data points \[(x_1,y_1),...(x_n,y_n)\]  \[if\ x_1<x_2<...<x_n\] then there exist a unique interpolating polynomial \[y=a_
{n-1}*x^{n-1}+a_{n-2}*x^{n-2}+a_{n-3}*x^{n-3}+...+a_0  \] with degree less than or equal to (n-1),this provides one condition that no two x are equal in our set of data points.
\subsection{Linear interpolation}\hrule
$\downarrow$\\
Linear Interpolation involves Interpolating Function f(x) to be a Straight Line between two interpolating points.For a value of x in the interval $(x_0,x_1)$, the value of y is given from the equation              which can interpreted using general form of a Straight Line.
So, one can call it as a weighted average of two data points, too.
 For a given Set of Data points $(x_0,y_0),(x_1,y_1),...,(x_n,y_n)$, we can Concatenate the interpolants between each pair of data points. Thus, we've a Continuous Function f(x), but Discontinuous at each data points.

\section{\textbf{Lagrange Interpolating Polynomial}}
\subsection{What is Lagrange interpoaltion}
The interpolating polynomial of least degree is unique,and lagrange formulae for interpolating curves ensure that the curve is of at least two degrees.
\subsection{Interpoaltion Formula}
$\Longrightarrow$ for k+1 data points:

\[L(x)=\sum\limits^k_{j=0}{y_j}*{l_j}(x)\] \[{l_j}(x)=\\\prod_{0<=m<=k}\dfrac{x-x_m}{x_j-x_m}\]
\[L(x)= \sum\limits^k_{j=0}{y_j}*\prod_{0<=m<=k}\dfrac{x-x_m}{x_j-x_m}\]
\[L(x)= y_0*\dfrac{x-x_1}{x_0-x_1}\dfrac{x-x_2}{x_0-x_2}\dfrac{x-x_3}{x_0-x_3}\dotsb\dfrac{x-x_k}{x_0-x_k}\]\[+y_1*\dfrac{x-x_0}{x_1-x_0}\dfrac{x-x_2}{x_1-x_2}\dfrac{x-x_3}{x_1-x_3}\dotsb\dfrac{x-x_k}{x_1-x_k}\]
\[+\dotsb y_k\dfrac{x-x_0}{x_k-x_0}*\dfrac{x-x_1}{x_k-x_1}\]\[\dotsb\dfrac{x-x_{k-1}}{x_k-x_{k-1}}\]
$\Longrightarrow$ If x=x1 means writing the same point in the function we are interpolating then \[L(x)=y_i\].

\section{Newton's Interpolation}
We are interested in finding out a polynomial which interpolates a pair of (x,y).Since we know that a polynomial of degree n can pass through n+1 points ,so we assume a polynomial of degree P(x) which interpolates \[(x_0,y_0),(x_1,y_1)\dotsb(x_n,y_n)\]Meanwhile,we can say that\[P_{n-1}(x) \text{passes through} (x_0,y_0),(x_1,y_1)\dotsb(x_{n-1},y_{n-1})\] Here \[P_n(x) \\and P_{n-1}(x) {\text{agrees at}} x_0,x_1\dotsb,x_{n-1}\]means
\[ P_n(x_0)=P_{n-1}(x_0)\]\[\vdots\]
\[P_n(x_{n-1})=P_{n-1}(x_{n-1})\]The difference of these two polnomial is a polynomial of at most degree n.
So we can write:\[P_n(x)=P_{n-1}(x)+G_n(x)  \dot{................} 1 \]from the above equation G(x) can't be a constant,hence can only be at the form\[G_n(x)=a_n(x-x_0)(x-x_1)(x-x_2)\dotsb(x-x_n)\]
from 1 \[G_n(x)=P_n(x)-P_{n-1}(x)\] \[a_n=\dfrac{P_n(x)-P_{n-1}(x)}{(x-x_0)(x-x_1)\dotsb(x-x_{n-1})}\] here \[a_n \text{is constant}\] to find out the value of constant x can be any x from our interpolated data set so,x can be replaced by \[a_n \text{which is interpolated by} P_n \text{but not} P_{n-1}\] So,
\[a_n=\dfrac{P_n(x_n)-P_{n-1}(x_n)}{(x_n-x_0)(x_n-x_1)\dotsb(x_n-x_{n-1})}\] now for n=1 \[a_1=\dfrac{P_n(x_1)-P_(n-1)(x_1)}{x_1-x_0}\] \[a_1=\dfrac{P_1(x_1)-P_0(x_1)}{x_1-x_0} \] here P(x) is a constant polynomial or in other words it interpolates at only one point ,i.e.\[(x_0,y_0)\] So \[a_1=\dfrac{P_1(x_1)-(y_0)}{(x_1)-(x_0)}\]here\[P_1(x_1)=y_1\] Divided difference between two points\[a_1=\underbrace{\dfrac{y_1-y_0}{x_1-x_0}}=F(x_1,x_0)\]\[P_1(x)=P_0+a_1(x-x_0)\]\[P_1(x)=F_0+a_1(x-x_0)\]Similarly for n=2
\[a_2=\dfrac{P_2(x_2)-P_1(x_2)}{(x_2-x_0)(x_2-x_1)}\] \[P_1(x_2)=F_0+a_1(x_2-x_0)\]So,
\[a_2=\dfrac{P_2(x_2)-F_0-a_1(x_2-x_0)}{(x_2-x_0)(x_2-x_1)}\]\[=\dfrac{y_2-F_0-a_1(x_2-x_0)}{x_2-x_0}\]On evaluating the equation \[a_2=F[x_0,x_1,x_2]\dotsb \text{Divided difference}\] So
\[P_2(x)=P_1(x)+F[x_0,x_1,x_2](x-x_0)(x-x_1)\]hence,in general form \[\boxed{P_k(x)=P_{k-1}+F[x_0,x_1\dotsb,x_k](x-x_0)(x-x_1)(x-x_k)}\]On recursively solving this gives \[P_n(x=F_0+(x-x_0)F[x-x_0]+(x-x_0)(x-x_1)F[x_0,x_1,x_2]\]\[+\dotsb+(x-x_0)(x-x_1)\dotsb(x-x_{n-1}F[x_0,x_1,x_2 \dotsb x_n]\]

\newpage
\section{Present Development Over Project}\hrule
$\downarrow$\\
%%%%%%%%%% INFORMATION %%%%%%%%%%

Till now we are working over the field of digital image processing ,which includes,image magnification and geometrical transformations.
\\
\subsection{Image Magnification}\hrule
$\downarrow$ \\
Stepwise Development:\\
\begin{itemize}
\item Fetching a sample digital image.\\
\begin{figure}[h]
\begin{center}$
\begin{array}{cc}
\includegraphics[width=2.5in]{sample.jpeg} &
\includegraphics[width=3.5in]{sample1.png}
\end{array}$
\end{center}
\caption{Sample image \& its respective Data sheet}
\end{figure}
\newpage
\item Doubling the order of the matrice and moving all the initial data points to even order points,keeping all other remaining data points empty.
\\
\begin{figure}[htbp]
\centering

  \includegraphics[width=1.015\linewidth]{intermediate.png}
  \caption{Data Sheet After Scaling}
  \label{fig:sub1}

\end{figure}


\item Our aim is to fill the remaining data points using linear interpolation . What we do is to fill each row using linear interpolation and then each column with linear interpolation so the value of A is weighted average of  1 \& 2 hence,\\
\begin{center}
A=[value(1)+value(2)]/2
\end{center} 
The general equation representing linear interpolation is$\rightarrow$
	\[c=\lambda(c_1)+1-\lambda(c_2)\] here,
\[\lambda\rightarrow parameter\] \[c=value[A]\] and \[c_1=value[1]\] \\
\begin{figure}[h]
\begin{center}$
\begin{array}{cc}
\includegraphics[width=3.05in]{interpolated.jpeg} &
\includegraphics[width=3.05in]{data2.png}
\end{array}$
\end{center}
\caption{Sample image \& its respective Data sheet After Interpolation}
\end{figure}

therefore,the general equation representing linear interpolation is $\rightarrow$
\item Now the new data point generated contains the properties of both data points.
\item Also the resulting image is of 2x of the previous sample image in both dimensions.\\
\begin{figure}[htbp]
\centering
\includegraphics[scale=0.5]{interpolated.jpeg}
\caption{Resulting Image}

\end{figure}
\end{itemize}
\newpage
\begin{center}
\section{\textbf{Reasearch And References }}\hrule
\begin{itemize}
\item 
\end{itemize}
\end{center}
\end{document}
